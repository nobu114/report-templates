\documentclass[10.5pt, a4paper, titlepage]{ltjsarticle}
\usepackage{luatexja}
\usepackage{geometry}
\usepackage{graphicx}
\usepackage{float}
% ソースコードを貼るための設定
\lstset{
  basicstyle={\ttfamily},
  columns=fixed,
  basewidth=0.5em,
  numbers=left,
  frame=lines
}
% フォントを選択
\usepackage[hiragino-pro]{luatexja-preset}
\geometry{margin=2cm}
% 1ページあたりの行数と1行あたりの文字数の指定のための定義
\makeatletter
\def\mojiparline#1{
\newcounter{mpl}
\setcounter{mpl}{#1}
\@tempdima=\linewidth
\advance\@tempdima by-\value{mpl}zw
\addtocounter{mpl}{-1}
\divide\@tempdima by \value{mpl}
\advance\kanjiskip by\@tempdima
\advance\parindent by\@tempdima
}
\makeatother
\def\linesparpage#1{
\baselineskip=\textheight
\divide\baselineskip by #1
}
\renewcommand{\lstlistingname}{ソースコード}
\title{タイトル}
\author{学籍番号 名前 \\ メールアドレス}
\date{}
\begin{document}
% \mojiparline{1} %X:一行あたり文字数の指定
% \linesparpage{40} %Y:1ページあたり行数の指定
\maketitle
\newpage

\tableofcontents
\clearpage

\section{はじめに}
% lstinputlisting[caption=hogehoge.py]{src/hogehoge.py}
% \begin{figure}[H]
%     \includegraphics[width=15cm]{hogehoge.png}
%     \caption{説明}
%     \centering
% \end{figure}
\section{章1}

\section{章2}

\section{まとめ}
\section*{参考文献}
\begin{enumerate}
    \item 参考文献
\end{enumerate}
\end{document}
